\documentclass[svgnames]{beamer}
%%\documentclass[svgnames,handout]{beamer}

%\usepackage[utf8]{inputenc}
\usepackage[T1]{fontenc}
\usepackage{graphicx}
\usepackage{longtable}
\usepackage{wrapfig}
\usepackage{rotating}
\usepackage[normalem]{ulem}
\usepackage{amsmath}
\usepackage{amssymb}
\usepackage{capt-of}
%\usepackage{hyperref}
\mode<beamer>{\usetheme{Darmstadt}}

%%--------------------------------------------------------------------------------
%%\usepackage[svgnames]{xcolor}
\usepackage{mathrsfs}
\usepackage{tikz-cd}

\usepackage{fontspec}
\setmonofont{FreeMono}
\setmainfont{FreeSerif}

\usepackage{unicode-math}

\usepackage{amsthm}
\usepackage{thmtools}
%\usepackage{cleveref}

%\usepackage[cachedir=mintedcache]{minted}
\usepackage{minted}
\usemintedstyle{tango}
\setminted[bash]{bgcolor=NavajoWhite}
\setminted[output]{bgcolor=NavajoWhite}
\setminted[python]{bgcolor=Lavender}

\newmintinline[lean]{lean4}{bgcolor=lavender}
\newminted[leancode]{lean4}{fontsize=\footnotesize,bgcolor=Lavender}
\setminted[lean]{bgcolor=LightBlue}

\usepackage{newunicodechar}
\newfontfamily{\freeserif}{DejaVu Sans}
\newunicodechar{✝}{\freeserif{✝}}
\newunicodechar{∀}{\ensuremath{\forall}}
\newunicodechar{→}{\ensuremath{\to}}
\newunicodechar{≤}{\ensuremath{\le}}
\newunicodechar{⧸}{/}


\newcommand{\Z}{\mathbf{Z}}
\newcommand{\Q}{\mathbf{Q}}
\newcommand{\R}{\mathbf{R}}
\newcommand{\C}{\mathbf{C}}
\newcommand{\F}{\mathbf{F}}
\newcommand{\N}{\mathbf{N}}

\newcommand{\LL}{\mathscr{L}}
\newcommand{\pp}{\mathbf{p}}
\newcommand{\xx}{\mathbf{x}}
\newcommand{\yy}{\mathbf{y}}
\newcommand{\vv}{\mathbf{v}}
\newcommand{\ww}{\mathbf{w}}
%%--------------------------------------------------------------------------------
\author{George McNinch}
\date{2025-06-29 14:50:51 EDT (george@valhalla)}
\title{Demo slides}
\hypersetup{
 pdfauthor={George McNinch},
 pdftitle={Demo slides},
 pdfkeywords={modelling},
 pdfsubject={},
 pdfcreator={Emacs 31.0.50 (Org mode 9.7.11)}, 
 pdflang={English}}
\usepackage{biblatex}

\begin{document}

\maketitle
\begin{frame}{Outline}
\tableofcontents
\end{frame}

\section{Section one}
\label{sec:one}
\begin{frame}[label={sec:examples},fragile]{Lean examples}
 \begin{itemize}[<+->]
\item example: defn of convergence of a sequence

\begin{minted}[]{lean}
-- definition of "u tends to ℓ" 
def seq_limit (u : ℕ → ℝ) (ℓ : ℝ) :=
  ∀ ε > 0, ∃ N, ∀ n ≥ N, |u n - ℓ| ≤ ε
\end{minted}
\end{itemize}
\end{frame}
\begin{frame}[label={sec:appending-lists},fragile]{appending lists}
 \begin{itemize}[<+->]
\item Here is some \texttt{Lean} code that \emph{appends two lists}.
\begin{minted}[]{lean}
def append {α:Type} (xs ys : List α)
  : List a :=
  match xs with
  | [] => ys
  | z :: zs => z :: append zs ys
\end{minted}

\item e.g.
\begin{minted}[]{lean}
append ["a", "b", "c"] ["d", "e"]
\end{minted}
evaluates to
\begin{minted}[]{lean}
["a", "b", "c", "d", "e"] : List String
\end{minted}
\end{itemize}
\end{frame}
\section{Algebra}
\label{sec:"finite-algebra"}
\begin{frame}[label={sec:vector-spaces}]{vector spaces}
\begin{itemize}[<+->]
\item Let \(k\) be a \alert{field}.

\item and let \(V\) be a finite dimensional vector space over \(k\).

\item Let \(\beta:V \times V \to k\) be a \alert{bilinear form}

\item and suppose that \(\beta\) is \alert{nondegenerate} and \alert{symmetric}

\item (also suppose for convenience that \(p\neq 2\) where \(p\) is the characteristic of \(k\))
\end{itemize}
\end{frame}
\end{document}
